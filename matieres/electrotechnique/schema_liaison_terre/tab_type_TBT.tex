%--------------------------------------
%ELECTROTECHNIQUE - SCHEMA DE LIAISON A LA TERRE
%--------------------------------------

%utiliser les environnement \begin{comment} \end{comment} pour mettre en commentaire le préambule une fois la programmation appelée dans le document maître (!ne pas oublier de mettre en commentaire \end{document}!)

\begin{comment}

\documentclass[a4paper, 11pt, twoside, fleqn]{memoir}

\usepackage{AOCDTF}

%--------------------------------------
%CANEVAS
%--------------------------------------

\newcommand\BoxColor{\ifcase\thechapshift blue!30\or brown!30\or pink!30\or cyan!30\or green!30\or teal!30\or purple!30\or red!30\or olive!30\or orange!30\or lime!30\or gray!\or magenta!30\else yellow!30\fi} %définition de la couleur des marqueurs de chapitre

\newcounter{chapshift} %compteur de chapitre du marqueur de chapitre
\addtocounter{chapshift}{-1}
	
\newif\ifFrame %instruction conditionnelle pour les couleurs des pages
\Frametrue

\pagestyle{plain}

% the main command; the mandatory argument sets the color of the vertical box
\newcommand\ChapFrame{%
\AddEverypageHook{%
\ifFrame
\ifthenelse{\isodd{\value{page}}}
  {\backgroundsetup{contents={%
  \begin{tikzpicture}[overlay,remember picture]
  \node[
  	rounded corners=3pt,
    fill=\BoxColor,
    inner sep=0pt,
    rectangle,
    text width=1.3cm,
    text height=5.5cm,
    align=center,
    anchor=north west
  ] 
  at ($ (current page.north west) + (-0cm,-2*\thechapshift cm) $) %nombre négatif = espacement des marqueurs entre les différents chapitres (à régler en fin de rédaction) (4.5cm vaut un espacement équivalement à la hauteur du marqueur, une page peut en contenir 6 avec cet espacement-la mais il est le plus équilibré)
    {\rotatebox{90}{\hspace*{.3cm}%
      \parbox[c][0.9cm][t]{5cm}{%
        \raggedright\textcolor{black}{\sffamily\textbf{\leftmark}}}}};
  \end{tikzpicture}}}
  }
  {\backgroundsetup{contents={%
  \begin{tikzpicture}[overlay,remember picture]
  \node[
  	rounded corners=3pt,
    fill=\BoxColor,
    inner sep=0pt,
    rectangle,
    text width=1.3cm,
    text height=5.5cm,
    align=center,
    anchor=north east
  ] 
  at ($ (current page.north east) + (-0cm,-2*\thechapshift cm) $) %nombre négatif = espacement des marqueurs entre les différents chapitres (à régler en fin de rédaction) (4.5cm vaut un espacement équivalement à la hauteur du marqueur, une page peut en contenir 6 avec cet espacement-la mais il est le plus équilibré)
    {\rotatebox{90}{\hspace*{.3cm}%
      \parbox[c][0.9cm][t]{5cm}{%
        \raggedright\textcolor{black}{\sffamily\textbf{\leftmark}}}}};
  \end{tikzpicture}}}%
  }
  \BgMaterial%
  \fi%
}%
  \stepcounter{chapshift}
}

\renewcommand\chaptermark[1]{\markboth{\thechapter.~#1}{}} %redéfinition du marqueur de chapitre pour ne contenir que le titre du chapitre %à personnaliser selon le nombre de chapitre dans le cours

%--------------------------------------
%corps du document
%--------------------------------------

\begin{document} %corps du document
	\openleft %début de chapitre à gauche

\end{comment}

\begin{landscape}
\begin{table}
\caption{Types de Très Basse Tension\label{tab:type_TBT}}
\begin{tabularx}{\linewidth}{p{3,5cm} X p{3cm} p{3,5cm} p{3,5cm} p{3,5cm} p{2cm}}
\toprule
\thead{Domaine \\de tension}		& \thead{Alimentation}			& \thead{Liaison à\\ la terre}			& \thead{Sectionnement et\\protection contre\\ les court-circuits}		& \thead{Protection contre \\ les contacts\\ indirects}		& \thead{Protection contre\\ les contacts\\ directs}				& \thead{Récepteur} \\
\midrule
TBTS (Très Basse Tension de Sécurité)		& Transformateur de sécurité conforme à la norme NF C 52 742		& \makecell[c]{Interdite}		& De tous des conducteurs actifs		& \makecell[c]{Non}		& \makecell[c]{Non} 	& \\
\addlinespace

															& \multicolumn{6}{l}{%
\begin{circuitikz}[circuit ee IEC relay]
%\DrawGrid{(0,-0.5)}{(21,0.5)} %grille d'aide pour le placement des objets
\draw (20.5,0) node[draw](Z){Z};
\draw (Z.west) node[circ, left](E){};
\draw (0,0) to[oosourcetrans, l=classe II] (3,0)
to[make contact={point left, circuit breaker={point left}, thick, name=dis}] (16,0)
to (E);
\draw [thick] (1.7,0.4) arc (90:-90:0.4) -- (0.9,-0.4) -- (0.9,0.4) -- (1.7,0.4);
\draw [thick] (1.5,0.3) -- (1.5,-0.3);
\end{circuitikz}} \\
\addlinespace[1cm]

TBTP (Très Basse Tension de Protection)		& Transformateur de sécurité conforme à la norme NF C 52 742		&  Conducteur actif relié à la terre		& De tous des conducteurs actifs		& \makecell[c]{Non}		& \makecell[c]{Non} 	& \\
\addlinespace
															& \multicolumn{6}{l}{%
\begin{circuitikz}[circuit ee IEC relay]
%\DrawGrid{(0,-0.5)}{(21,0.5)} %grille d'aide pour le placement des objets
\draw (20.5,0) node[draw](Z){Z};
\draw (Z.west) node[circ, left](E){};
\draw (0,0) to[oosourcetrans, l=classe I] (3,0)
to node[circ](C){} (8,0)
to[make contact={point left, circuit breaker={point left}, thick, name=dis}] (11,0)
to (E);
\draw (C) -- (5.5,-1) node[pground]{};
\draw [thick] (1.5,0.3) -- (1.5,-0.3);
\end{circuitikz}} \\
\addlinespace[1cm]

TBTF (Très Basse Tension de Fonctionnelle)		& Transformateur de sécurité d'origine indéterminée		&  Conducteur actif relié à la terre		& De tous des conducteurs actifs		& \makecell[c]{Oui (DDR)}		& \makecell[c]{Oui (appareil IP2X)} 	& \\
\addlinespace
															& \multicolumn{6}{l}{%
\begin{circuitikz}[circuit ee IEC relay]
%\DrawGrid{(0,-0.5)}{(21,0.5)} %grille d'aide pour le placement des objets
\draw (19.5,0) node[draw](Z){Z};
\draw (Z.west) node[circ, left](E){};
\draw (Z.east) node[circ, right](F){};
\draw [rounded corners] (F) -| (20.5,-1) node [tlground]{};
\draw (0,0) to[oosourcetrans] (3,0)
to node[circ](C){} (8,0)
to[make contact={point left, circuit breaker={point left}, thick, name=dis}] (11,0)
to node[iloopshape, rotate=180](D){} (16,0)
to (E);
\draw [dashed, rounded corners] (dis.mid) to (9.5,-1) -- (13.5,-1) to (D.i);
\draw (C) -- (5.5,-1) node[pground]{} ;
\end{circuitikz}} \\
\addlinespace[1cm]

\bottomrule
\end{tabularx}

\end{table}

\end{landscape}

%\end{document}

