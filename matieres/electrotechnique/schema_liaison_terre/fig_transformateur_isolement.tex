%--------------------------------------
%ELECTROTECHNIQUE - SCHEMA DE LIAISON A LA TERRE
%--------------------------------------

%utiliser les environnement \begin{comment} \end{comment} pour mettre en commentaire le préambule une fois la programmation appelée dans le document maître (!ne pas oublier de mettre en commentaire \end{document}!)

\begin{comment}

\documentclass[a4paper, 11pt, twoside, fleqn]{memoir}

\usepackage{AOCDTF}

%--------------------------------------
%CANEVAS
%--------------------------------------

\newcommand\BoxColor{\ifcase\thechapshift blue!30\or brown!30\or pink!30\or cyan!30\or green!30\or teal!30\or purple!30\or red!30\or olive!30\or orange!30\or lime!30\or gray!\or magenta!30\else yellow!30\fi} %définition de la couleur des marqueurs de chapitre

\newcounter{chapshift} %compteur de chapitre du marqueur de chapitre
\addtocounter{chapshift}{-1}
	
\newif\ifFrame %instruction conditionnelle pour les couleurs des pages
\Frametrue

\pagestyle{plain}

% the main command; the mandatory argument sets the color of the vertical box
\newcommand\ChapFrame{%
\AddEverypageHook{%
\ifFrame
\ifthenelse{\isodd{\value{page}}}
  {\backgroundsetup{contents={%
  \begin{tikzpicture}[overlay,remember picture]
  \node[
  	rounded corners=3pt,
    fill=\BoxColor,
    inner sep=0pt,
    rectangle,
    text width=1.3cm,
    text height=5.5cm,
    align=center,
    anchor=north west
  ] 
  at ($ (current page.north west) + (-0cm,-2*\thechapshift cm) $) %nombre négatif = espacement des marqueurs entre les différents chapitres (à régler en fin de rédaction) (4.5cm vaut un espacement équivalement à la hauteur du marqueur, une page peut en contenir 6 avec cet espacement-la mais il est le plus équilibré)
    {\rotatebox{90}{\hspace*{.3cm}%
      \parbox[c][0.9cm][t]{5cm}{%
        \raggedright\textcolor{black}{\sffamily\textbf{\leftmark}}}}};
  \end{tikzpicture}}}
  }
  {\backgroundsetup{contents={%
  \begin{tikzpicture}[overlay,remember picture]
  \node[
  	rounded corners=3pt,
    fill=\BoxColor,
    inner sep=0pt,
    rectangle,
    text width=1.3cm,
    text height=5.5cm,
    align=center,
    anchor=north east
  ] 
  at ($ (current page.north east) + (-0cm,-2*\thechapshift cm) $) %nombre négatif = espacement des marqueurs entre les différents chapitres (à régler en fin de rédaction) (4.5cm vaut un espacement équivalement à la hauteur du marqueur, une page peut en contenir 6 avec cet espacement-la mais il est le plus équilibré)
    {\rotatebox{90}{\hspace*{.3cm}%
      \parbox[c][0.9cm][t]{5cm}{%
        \raggedright\textcolor{black}{\sffamily\textbf{\leftmark}}}}};
  \end{tikzpicture}}}%
  }
  \BgMaterial%
  \fi%
}%
  \stepcounter{chapshift}
}

\renewcommand\chaptermark[1]{\markboth{\thechapter.~#1}{}} %redéfinition du marqueur de chapitre pour ne contenir que le titre du chapitre %à personnaliser selon le nombre de chapitre dans le cours

%--------------------------------------
%corps du document
%--------------------------------------

\begin{document} %corps du document
	\openleft %début de chapitre à gauche

\end{comment}

\begin{wrapfigure}{R}{0pt} %insertion figure dans texte
\begin{circuitikz}[circuit ee IEC]
%\DrawGrid{(-1,-5)}{(7,3)} %grille d'aide pour le placement des objets

\fill [gray!50] (-1,-3.5) -- (5,-3.5) -- (5,-3.7) -- (-1,-3.7) -- cycle;
\draw [thick] (-1,-3.5) -- (5,-3.5);

\node (T1) [oosourcetransshape,prim=delta,sec=wye] at (0,0) {};
\draw [brown] (-1,0.3) to (-0.5,0.3) to node {} (T1.prim1);
\draw [black] (-1,0) to (-0.5,0) to node {} (T1.prim2);
\draw [gray] (-1,-0.3) to (-0.5,-0.3) to node {} (T1.prim3);
\draw [brown] (5,0.3) to (1,0.3) to (0.5,0.3) to node {} (T1.sec1);
\draw [black] (5,0.1) to (1,0.1) to (0.5,0.1) to node {} (T1.sec2);
\draw [gray] (5,-0.1) to (1,-0.1) to (0.5,-0.1) to node {} (T1.sec3);
\draw [blue] (5,-0.3) to (1,-0.3) to (0.5,-0.3) to node {} (T1.sec4);
\node (G) [tlground] at (0,-3.9) {};
\draw [green!] (G) to (0,-0.4) to node {} (T1.sec4) ; 
\draw [dashed, yellow!] (G) to (0,-0.4) to node {} (T1.sec4) ;
\node (G) [tlground] at (0,-3.9) {};
\node (T1) [oosourcetransshape,prim=delta,sec=wye] at (0,0) {};

\node (R) [resistor, point down, tiny circuit symbols] at (1.7,-2.2) {};

\node (T2) [oosourcetransshape, rotate=-90] at (1.5,-1) {};
\draw [brown] (1.7,0.3) to (1.7,-0.5) to node {} (T2.prim1');
\draw [blue] (1.3,-0.3) to (1.3,-0.5) to node {} (T2.prim2');
\draw [blue] (R) to (1.7,-2.9) to (1.3,-2.9) to (1.3,-1.5) to node {} (T2.sec2');
\draw [brown] (R) to (1.7,-1.5) to node {} (T2.sec1');
\node (T2) [oosourcetransshape, rotate=-90] at (1.5,-1) {};
\draw [thick] (1.2,-1) -- (1.8,-1);


\draw (1.7,0.3) node[circ, scale=0.5]{};
\draw (1.3,-0.3) node[circ, scale=0.5]{};

\draw (3,-1.5) -- (3.3,-2.5) -- (3.6,-1.5) ; %tronc
\draw (3.3,-1.5) -- (3.3, -1.3); %cou
\draw (3.3,-1) circle [radius=0.3cm]; %tête
\draw (1.7,-1.7) -- (1.9,-1.7) -- (2.4,-1.4)  -- (3,-1.5) -- (3.6,-1.5) -- (4,-2) -- (4,-2.6) -- (3.9,-2.8); %bras
\draw (2.8,-3.3) -- (3,-3.4) -- (3.1, -2.9) -- (3.3,-2.5) -- (3.6,-3) -- (3.6,-3.4) -- (3.4,-3.5); %jambes
\filldraw ([shift=(-10:0.3cm)]3.3,-1) arc (-10:150:0.3cm); %casquette
\draw (3.04,-0.84) -- ++ (140:0.3cm); 

%\fill [yellow!, decoration=lightning bolt, decorate] (1.7,-1.7) -- ++ (0.5,0.8); %éclairs
\path [postaction={on each segment={mid arrow=red}}] node {} (1.7,-1.5) -- (1.7,-1.7) -- (1.9,-1.7) -- (2.4,-1.4)  -- (3,-1.5) -- (3.6,-1.5) -- (4,-2) -- (4,-2.6) -- (3.9,-2.8); 
\path [postaction={on each segment={mid arrow=red}}] (1.7,-2.5) -- (1.7,-2.9) -- (1.3,-2.9) -- (1.3,-1.5) to node {} (T2.sec2'); 

\end{circuitikz}
\end{wrapfigure}

%\end{document}

