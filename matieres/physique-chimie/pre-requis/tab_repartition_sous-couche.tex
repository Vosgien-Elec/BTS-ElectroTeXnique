%utiliser les environnement \begin{comment} \end{comment} pour mettre en commentaire le préambule une fois la programmation appelée dans le document maître (!ne pas oublier de mettre en commentaire \end{document}!)

\begin{comment}

\documentclass[a4paper, 11pt, twoside, fleqn]{memoir}

\usepackage{AOCDTF}

%--------------------------------------
%CANEVAS
%--------------------------------------

\newcommand\BoxColor{\ifcase\thechapshift blue!30\or brown!30\or pink!30\or cyan!30\or green!30\or teal!30\or purple!30\or red!30\or olive!30\or orange!30\or lime!30\or gray!\or magenta!30\else yellow!30\fi} %définition de la couleur des marqueurs de chapitre

\newcounter{chapshift} %compteur de chapitre du marqueur de chapitre
\addtocounter{chapshift}{-1}
	
\newif\ifFrame %instruction conditionnelle pour les couleurs des pages
\Frametrue

\pagestyle{plain}

% the main command; the mandatory argument sets the color of the vertical box
\newcommand\ChapFrame{%
\AddEverypageHook{%
\ifFrame
\ifthenelse{\isodd{\value{page}}}
  {\backgroundsetup{contents={%
  \begin{tikzpicture}[overlay,remember picture]
  \node[
  	rounded corners=3pt,
    fill=\BoxColor,
    inner sep=0pt,
    rectangle,
    text width=1.3cm,
    text height=5.5cm,
    align=center,
    anchor=north west
  ] 
  at ($ (current page.north west) + (-0cm,-2*\thechapshift cm) $) %nombre négatif = espacement des marqueurs entre les différents chapitres (à régler en fin de rédaction) (4.5cm vaut un espacement équivalement à la hauteur du marqueur, une page peut en contenir 6 avec cet espacement-la mais il est le plus équilibré)
    {\rotatebox{90}{\hspace*{.3cm}%
      \parbox[c][0.9cm][t]{5cm}{%
        \raggedright\textcolor{black}{\sffamily\textbf{\leftmark}}}}};
  \end{tikzpicture}}}
  }
  {\backgroundsetup{contents={%
  \begin{tikzpicture}[overlay,remember picture]
  \node[
  	rounded corners=3pt,
    fill=\BoxColor,
    inner sep=0pt,
    rectangle,
    text width=1.3cm,
    text height=5.5cm,
    align=center,
    anchor=north east
  ] 
  at ($ (current page.north east) + (-0cm,-2*\thechapshift cm) $) %nombre négatif = espacement des marqueurs entre les différents chapitres (à régler en fin de rédaction) (4.5cm vaut un espacement équivalement à la hauteur du marqueur, une page peut en contenir 6 avec cet espacement-la mais il est le plus équilibré)
    {\rotatebox{90}{\hspace*{.3cm}%
      \parbox[c][0.9cm][t]{5cm}{%
        \raggedright\textcolor{black}{\sffamily\textbf{\leftmark}}}}};
  \end{tikzpicture}}}%
  }
  \BgMaterial%
  \fi%
}%
  \stepcounter{chapshift}
}

\renewcommand\chaptermark[1]{\markboth{\thechapter.~#1}{}} %redéfinition du marqueur de chapitre pour ne contenir que le titre du chapitre %à personnaliser selon le nombre de chapitre dans le cours

%--------------------------------------
%corps du document
%--------------------------------------

\begin{document} %corps du document
	\openleft %début de chapitre à gauche

\end{comment}



\begin{wrapfigure}{r}{0pt} %insertion figure dans texte
\begin{tikzpicture}[font=\sffamily] 
\matrix (mat) [matrix of nodes,column sep=0.2cm,row sep=2mm,
 nodes={align=left},nodes in empty cells,
 row 1/.append style={nodes={font=\footnotesize\sffamily}},
 ]{
 & 2 & 6 & 10 & 14 \\
 & 1s & & & \\
 & 2s & 2p & & \\
 & 3s & 3p & 3d & \\
 & 4s & 4p & 4d & 4f\\
 & 5s & 5p & 5d & 5f\\
 & 6s & 6p & 6d & \\
 & 7s & 7p & 7d & \\
 };
 \path[name path=boundary] (mat-1-1.south) -- (mat-1-3.south)
 to[bend right] (mat-2-4.south) to[bend left] ([xshift=6mm]mat-5-5.south)
  -- (mat.south east);
 \foreach \X [remember=\X as \LastX] in {2,...,10}
 {\ifnum\X<8
    \path[overlay,name path=\X-line] 
    ($(mat-\X-2.south west)-0.1*($(mat-2-5.south west)- (mat-6-1.south west)$)$) coordinate(\X-start) 
    -- ++ ($5*($(mat-2-5.south west)- (mat-6-1.south west)$)$);
  \else
    \pgfmathtruncatemacro{\Xprime}{\X-6}
    \path[overlay,name path=\X-line] 
    ($(mat-8-\Xprime.south west)-0.1*($(mat-2-5.south west)- (mat-6-1.south west)$)$) 
    coordinate(\X-start) -- ++ ($5*($(mat-2-5.south west)- (mat-6-1.south west)$)$);
  \fi   
  \draw[blue!50,latex-,shorten <=2pt,name intersections={of=\X-line and boundary,by=i-\X}]
  (\X-start)--(i-\X);
  \ifnum\X>2
   \draw[blue!50,dashed,overlay] 
   let \p1=($(mat-2-5.south west)- (mat-6-1.south west)$),\n1={atan2(\y1,\x1)} in
   (i-\X) to[out=\n1,in=180+\n1] (\LastX-start);
  \fi
  }
\end{tikzpicture}
\caption{Ordre de répartition des sous-couches électroniques}
\label{fig:repartition_sous-couche}
\end{wrapfigure}

%\end{document} %corps du document
